% Created 2011-08-08 Mon 10:55
\documentclass[11pt]{article}
\usepackage[utf8]{inputenc}
\usepackage[T1]{fontenc}
\usepackage{fixltx2e}
\usepackage{graphicx}
\usepackage{longtable}
\usepackage{float}
\usepackage{wrapfig}
\usepackage{soul}
\usepackage{t1enc}
\usepackage{textcomp}
\usepackage{marvosym}
\usepackage{wasysym}
\usepackage{latexsym}
\usepackage{amssymb}
\usepackage{hyperref}
\tolerance=1000
\usepackage{color}
\usepackage{listings}
\providecommand{\alert}[1]{\textbf{#1}}
\begin{document}



\title{CPR Assignment}
\author{Giles Chamberlin}
\date{8 Aug 2011}
\maketitle



\pagebreak
\section{Why tail recursive?}
\label{sec-1}


The server is intended to stay operational for long periods of time
and many client calls.  The main server loop will therefore be entered
many times.  If that loop is not tail recursive each entry will put a
new stack frame on the call stack, eventually exhausting stack space.
If the function is tail recursive, tail call optimisation can reuse
the current frame, replacing the function call with a jump
instruction.  This conserves stack space and allows infinite
recursion.
\section{Why use ReferenceId instead of Pid?}
\label{sec-2}


The intention is that the client should be unaware of any failure in
the server.  Pids are tied to a single unique server instance and so
sending to a particular Pid will be invalidated if the server fails.
The traditional approach of adding a level of indirection allows the
client to refer to a ReferenceId which is mapped in turn to a Pid.
That mapping is opaque to the client.
\section{Upgrade process}
\label{sec-3}


For each node to be upgraded:

\begin{itemize}
\item Copy new beam files to node
\item Mark node as in maintenance mode.  No new carts should be created on
  this node.
\item Ideally:
\begin{itemize}
\item allow time to pass and wait for all carts to terminate naturally.
\item return node from maintenance mode.
\item next fully qualified call to the cart on this node will invoke the
    new code
\end{itemize}
\item Pragmatically:
\begin{itemize}
\item Some transactions will persist beyond a reasonable time scale for
    those transactions.
\item Wait as long as possible to minimise the number of active
    transactions.
\item Kill the remaining carts
\item Allow the backup process to take over
\item Once the node is quiet return from maintenance mode
\item next fully qualified call to the cart on this node will invoke the
     new code
\end{itemize}
\end{itemize}
\section{Synchronisation after server failure}
\label{sec-4}


As implemented the synchronisation strategy relies on the two servers
sharing an underlying DETS file.  Should one server fail, the other
will take over responsibility for reading and writing to that file.
Whilst this is robust in the presence of a software failure in the
server it is useless in the face of a network split or node failure.
If the backup server cannot access the DETS file then no continuation
is possible.

If each server were to own its own persistent store and transaction
information be shared between the two each time a transaction occurs
(transactional replication), then survival in the face of node failure
is possible: the backup simply takes over and, on restoration, the
original can simply copy all new transactions to its own store.

Things become more complicated in the event of a network split.  Both
nodes now take on the role of master and will accept state changing
transactions. When the network heals the system is left with two,
allegedly definitive, copies of the transaction state.  Combining
these two views (merge replication) depends on the nature of the
transactions that have occurred and the local business policy.  In
general it is not possible to guarantee a ``correct'' view of the state
of the system and operator intervention, or automated business rules,
may be required.  

\end{document}
